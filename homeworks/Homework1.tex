\documentclass[10pt]{beamer}

\usepackage{amssymb,amsmath}
\usepackage{graphicx}
\usepackage{url}
\usepackage{color}
\usepackage{pagenote}[continuous,page]
\usepackage{relsize}		% For \smaller
\usepackage{url}			% For \url
\usepackage{epstopdf}	% Included EPS files automatically converted to PDF to include with pdflatex

%For MindMaps
% \usepackage{tikz}%
% \usetikzlibrary{mindmap,trees,arrows}%

%%% Color Definitions %%%%%%%%%%%%%%%%%%%%%%%%%%%%%%%%%%%%%%%%%%%%%%%%%%%%%%%%%
%\definecolor{bordercol}{RGB}{40,40,40}
%\definecolor{headercol1}{RGB}{186,215,230}
%\definecolor{headercol2}{RGB}{80,80,80}
%\definecolor{headerfontcol}{RGB}{0,0,0}
%\definecolor{boxcolor}{RGB}{186,215,230}

%%% Save space in lists. Use this after the opening of the list %%%%%%%%%%%%%%%%
%\newcommand{\compresslist}{
%	\setlength{\itemsep}{1pt}
%	\setlength{\parskip}{0pt}
%	\setlength{\parsep}{0pt}
%}

%\setbeameroption{show notes on top}

% You should run 'pdflatex' TWICE, because of TOC issues.

% Rename this file.  A common temptation for first-time slide makers
% is to name it something like ``my_talk.tex'' or
% ``john_doe_talk.tex'' or even ``discrete_math_seminar_talk.tex''.
% You really won't like any of these titles the second time you give a
% talk.  Try naming your tex file something more descriptive, like
% ``riemann_hypothesis_short_proof_talk.tex''.  Even better (in case
% you recycle 99% of a talk, but still want to change a little, and
% retain copies of each), how about
% ``riemann_hypothesis_short_proof_MIT-Colloquium.2000-01-01.tex''?

\mode<presentation>
{
  \usetheme{CambridgeUS}
  \usecolortheme{dolphin}
  \useoutertheme{default}
  \useinnertheme{default}
  \setbeamercovered{invisible} % or whatever (possibly just delete it)
}
\beamertemplatenavigationsymbolsempty

\usepackage[english]{babel}
%\usepackage[latin1]{inputenc}
\usepackage{subfigure}

\usepackage{times}
\usepackage[T1]{fontenc}
\usepackage{CJKutf8}

%% makes the ppagenote command for figure references at the end.
\makepagenote
\renewcommand{\notenumintext}[1]{}
\newcommand{\ppagenote}[1]{\pagenote[Page \insertframenumber]{#1}}

\title[Experiment Design (01CH740)]{Experiment Design for Computer Sciences (01CH740)}
\author[Claus Aranha]{Claus Aranha\\{\footnotesize caranha@cs.tsukuba.ac.jp}}
\institute[U. Tsukuba]{University of Tsukuba, Department of Computer Sciences}



\title[]{Experiment Planning and Design}
\subtitle[]{Lecture 3-a: Homework Analysis}
\author[Claus Aranha]{Claus Aranha\\{\footnotesize caranha@cs.tsukuba.ac.jp}}
\institute{Department of Computer Science}

\begin{document}

\begin{frame}
  \maketitle
\end{frame}

\begin{frame}
  \frametitle{Outline}
  \begin{block}{}
    Make a description of your research topics, taking into account
    the concepts we discussed about ``Science'' and ``Experiment
    Design''
  \end{block}

  \begin{itemize}
    \item It is a good way to try to weed out some common mistakes;
    \item Will not be able to comment on everyone's work. Ask me
      questions after the class, or extrapolate to your own work.
    \item Remember that I am not an expert in every field -- these are
      generic scientific questions, more specific questions may apply
      in your field.
  \end{itemize}
\end{frame}

\begin{frame}
  \frametitle{Optics based quantum device}
  \begin{block}{What did I get from the description?}
  \begin{itemize}
  \item Description of a 2D simulator of quantum bit behavior
  \item Desire of the implementation of a 3D quantum bit behavior
  \end{itemize}
  \end{block}
  \begin{block}{My comments}
    \begin{itemize}
    \item How to we evaluate that the simulation is operating correctly?
    \item Exact behavior, or Statistical behavior?
    \item Minimum/Maximum treshold of behavior?
    \item What are the observable quantities in the simulation?
    \item \alert{a testing condition is necessary}
    \end{itemize}
  \end{block}
\end{frame}

\begin{frame}
  \frametitle{Image Analysis using Eigensolver clustering}
  \begin{block}{What did I get from the description?}
    \begin{itemize}
    \item Image segmentation - clustering of images in order to find features
    \item Use of the ``spectral Clustering'' measure
    \item Difficulty point - connecting theory and
      practice. \structure{I understand j00!}
    \end{itemize}
  \end{block}
  \begin{block}{My comments}
    \begin{itemize}
    \item Clustering is very easy to setup as an experiment!
    \item Cluster testing: NMI, Rand Index, other indexes
    \item Comparison with other clustering methods! (SVM, kmeans?)
    \item Do you want to focus on the method, or the application?
    \end{itemize}    
  \end{block}
\end{frame}

\begin{frame}
  \frametitle{Database Management Systems}
  \begin{block}{What did I get from the description?}
    \begin{itemize}
    \item Concerned about the confidentiality of data (when storing?
      when transmitting? when processing?)
    \item Improve performance of storage of encrypted data, but the
      database becomes ``complicated''
    \item \structure{Experiment}: execute SQL query and measure the
      time; Compare with execution time of existing (non encripted?)
      method;
    \end{itemize}
  \end{block}
  \begin{block}{My comments}
    \begin{itemize}
    \item ``Big Data has been the basis of attractive research fields''
      -- why is this attractive?
    \item Tradeoff is good, but what does ``complicated'' and
      ``performance'' mean?
    \item We could say that the goal is ``to perform encrypted
      operations at no additional performance burden''
    \item Need to be careful to define many experiment conditions
      (message size, message complexity, computer load, etc).
    \end{itemize}
  \end{block}
\end{frame}

\begin{frame}
  \frametitle{Virtual Machine research}
  \begin{block}{What did I get from the description?}
    \begin{itemize}
    \item Research about improvement of VM to use kernel instructions
    \item Topic involves some deep hacking of computer's memory
    \item System Benchmarks as experiments
    \end{itemize}
  \end{block}
  \begin{block}{My comments}
    \begin{itemize}
    \item The student identified different expectations from different
      results of his experiments.
    \item Experiments parameters and goals were well described
    \item This is a hard topic to perform experimentation, but the
      analysis of system performance is a good topic. Comparison with
      non-virtual systems is an obvious target.
    \end{itemize}
  \end{block}
\end{frame}

\begin{frame}
  \frametitle{Natural Language Processing}
  \begin{block}{What did I get from the description?}
    \begin{itemize}
    \item Statistical Machine translation;
    \item Different word alignment models;
    \item Analysis of ``data training'' perfomance.
    \end{itemize}
  \end{block}
  \begin{block}{My comments}
    \begin{itemize}
    \item Experiment is not well defined: ``data training'',
      ``performance'' are undefined terms.
    \item Do you mean correct translation rate?
    \item Or do you mean time?
    \item How do you measure both things?
    \item What other variations and uncertainties do you have to take
      into account?
    \end{itemize}
  \end{block}
\end{frame}

\begin{frame}
  \frametitle{Camera Controlled Robot}
  \begin{block}{What did I get from the description?}
    \begin{itemize}
    \item Automatic ``coating'' from camera data
    \item Decisions: How many cameras are needed? What kind of camera? What models?
    \item Testing: How to recover data from the camera? How to calibrate the camera?
    \item Conditions: Setup, light, error, etc/
    \end{itemize}
  \end{block}
  \begin{block}{My comments}
    \begin{itemize}
    \item Important distinction: tests for defining what are the best
      parts for a research model; tests for comparing the model with
      other research works;
    \item Model Validation vs Model development;
    \end{itemize}
  \end{block}
\end{frame}

\begin{frame}
  \frametitle{Secure Wireless Network}
  \begin{block}{What did I get from the description?}
    \begin{itemize}
    \item Detection of attackers/wiretapping in a wireless network
    \item Limited performance (power/communication/memory)
    \item Evaluation of data confidentiality
    \item Experiment with varied number of attacks
    \end{itemize}
  \end{block}
  \begin{block}{My comments}
    \begin{itemize}
    \item How about experiment without attacks? 
    \item The utility of controls;
    \item Problem: result is ``method is feasible'', which does not say much.
    \item ``High perfomance'', ``low performance'' as expected results
      can be quantified (we will do this in this course)
    \end{itemize}
  \end{block}
\end{frame}

\begin{frame}
  \frametitle{Stream Data Processing Engine}
  \begin{block}{What did I get from the description?}
    \begin{itemize}
    \item Many people use ``Performance'' with many different
      meanings;
    \item It is important to be specific in your word use to avoid
      misunderstandings;
    \end{itemize}
  \end{block}
\end{frame}

\begin{frame}
  \frametitle{Decentralized SNS development}
  \begin{block}{What did I get from the description?}
    \begin{itemize}
    \item Focus on data storage, data syncronization, data replication
    \item Difficulty to stablish P2P connection among peers with
      interfering situations
    \item Evaluation of application ``performance''
    \item Experiment: Time cost and data loss of communication over
      internet and LAN;
    \item Compare data loss and time cost against ``user tolerance''
    \item Stress test against multiple concurrent users;
    \end{itemize}
  \end{block}
\end{frame}

\begin{frame}
  \frametitle{High Performance Computation}
  \begin{block}{What did I get from the description?}
    \begin{itemize}
    \item Very Cool Title;
    \item High performance computing using GPU and MIC
    \item Application of TCA architecture to MIC
    \item Test of optimization implementations
    \item Is your research hard enough?
    \end{itemize}
  \end{block}
\end{frame}

\begin{frame}
  \frametitle{Intelligent Mobile Robot}
  \begin{block}{What did I get from the description?}
    \begin{itemize}
    \item Autonomous robot that can squeeze past crowds
    \item Algorithm is the critical part - how to control each wheel?
    \item Experiment: Test if robot can pass or not through crowd;
    \item Analysis of Noise
    \end{itemize}
  \end{block}
  \begin{block}{My comments}
    \begin{itemize}
    \item ``Experiment: pass through the crowd'' does not need to be a
      ``Yes/No'' experiment.
    \item How many times did you hit a person? (incremental improvement)
    \item What was the robot's time while evading people?
    \end{itemize}
  \end{block}
\end{frame}



\end{document}
